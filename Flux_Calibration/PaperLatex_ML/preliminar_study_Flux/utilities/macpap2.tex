\newcommand{\be}{\begin{equation}}
\newcommand{\ee}{\end{equation}}
\newcommand{\ben}{\begin{equation*}}
\newcommand{\een}{\end{equation*}}
\newcommand{\bean}{\begin{eqnarray}}
\newcommand{\eean}{\end{eqnarray}}
\newcommand{\bea}{\begin{eqnarray*}}
\newcommand{\eea}{\end{eqnarray*}}
\newcommand{\bfig}{\begin{figure}}
\newcommand{\efig}{\end{figure}}
\newcommand{\ba}{\begin{array}}
\newcommand{\ea}{\end{array}}

\newcommand{\gap}[3]{\mbox{\hspace{#1 ex} #2 \hspace{#3 ex}}}
%\newcommand{\nn}{\nonumber}
\newcommand{\hs}{\hspace}
\newcommand{\lb}{\label}
\newcommand{\ds}{\displaystyle}
\newcommand{\scst}{\scriptstyle}
\newcommand{\scsi}{\scriptsize}
\newcommand{\incg}{\includegraphics}

\newcommand{\na}{\nabla}
\newcommand{\pa}{\partial}
\newcommand{\Ra}{\Rightarrow \mbox{\hspace{1ex}}}
\newcommand{\ti}{\tilde}

\newcommand{\si}{\sigma}
\newcommand{\vsi}{\varsigma}
\newcommand{\ov}{\overline}
\newcommand{\ga}{\gamma}
\newcommand{\Ga}{\Gamma}
\newcommand{\eps}{\epsilon}
\newcommand{\ups}{\upsilon}
\newcommand{\Ups}{\Upsilon}
\newcommand{\var}{\varepsilon}
\newcommand{\ka}{\kappa}
\newcommand{\al}{\alpha}
\newcommand{\la}{\lambda}
\newcommand{\La}{\Lambda}
\newcommand{\de}{\delta}
\newcommand{\De}{\Delta}
\newcommand{\p}{\prime}
\newcommand{\ra}{\rightarrow}
\newcommand{\om}{\omega}
\newcommand{\Om}{\Omega}
\newcommand{\tha}{\theta}
\newcommand{\ze}{\zeta}
\newcommand{\bet}{\beta}
\newcommand{\bs}{\boldsymbol}
%%% Local Variables:
%%% mode: latex
%%% TeX-master: t
%%% End:


\DeclareSymbolFont{AMSb}{U}{msb}{m}{n}% or use amsfonts package
\DeclareMathSymbol{\C}{\mathalpha}{AMSb}{"43}
\DeclareMathSymbol{\R}{\mathalpha}{AMSb}{"52}
\DeclareMathSymbol{\K}{\mathalpha}{AMSb}{"4B}
\DeclareMathSymbol{\M}{\mathalpha}{AMSb}{"4D}
\DeclareMathSymbol{\N}{\mathalpha}{AMSb}{"4E}
\DeclareMathSymbol{\OO}{\mathalpha}{AMSb}{"4F}
\DeclareMathSymbol{\Ss}{\mathalpha}{AMSb}{"53}
\DeclareMathSymbol{\Q}{\mathalpha}{AMSb}{"51}
\DeclareMathSymbol{\Z}{\mathalpha}{AMSb}{"5A}
%\DeclareMathSymbol{\QM}{\mathalpha}{AMSb}{"3F}
\DeclareMathOperator{\diag}{diag} \DeclareMathOperator{\Div}{Div}
\DeclareMathOperator{\cof}{cof}
\DeclareMathOperator{\diverg}{div}\DeclareMathOperator{\diam}{diam}
\newcommand{\abs}[1]{\lvert#1\rvert}
\newcommand{\norm}[1]{\lVert#1\rVert}
\newcommand{\tverb}[1]{\text{\texttt{#1}}}
\newcommand{\sfrac}[2]{\genfrac{}{}{0 pt}{}{#1}{#2}}
\newcommand{\ud}{\mathrm{d}}
\newcommand{\figfont}{\footnotesize}

% Jump operator
\newcommand{\jl}{[\![}
\newcommand{\jr}{]\!]}

%\newtheorem{lemma}{Lemma}[section]
%\newtheorem{theorem}{Theorem}[section]
%\newtheorem{cor}{Corollary}[section]
%\newtheorem{proposition}{Proposition}[section]

\theoremstyle{definition}
\newtheorem{defi}{Definition}[section]

\theoremstyle{remark}
\newtheorem{remark}{Remark}[section]
\newtheorem*{notation}{Notation}

%\newtheorem{defi}{Definition}[section]
%\newtheorem{lemma}{Lemma}[section]
%\newtheorem{theorem}{Theorem}[section]
%\newtheorem{cor}{Corollary}[section]
%\newtheorem{remark}{Remark}[section]
%\newtheorem{proposition}{Proposition}[section]

\newcommand{\thmref}[1]{Theorem~\ref{#1}}
\newcommand{\secref}[1]{\S\ref{#1}}
\newcommand{\lemref}[1]{Lemma~\ref{#1}}

\newcommand{\bysame}{\mbox{\rule{3em}{.4pt}}\,}

\newcommand{\mnorm}[1]{%
  \left\vert\kern-0.9pt\left\vert\kern-0.9pt\left\vert #1
    \right\vert\kern-0.9pt\right\vert\kern-0.9pt\right\vert}

%\renewcommand{\crest}{\beltcrest}

\ifx\quick\undefined
\else
        \renewcommand{\incg}[2][]{\includegraphics[#1,draft=true]{#2}}       % Don't include anything in draft mode
\fi 