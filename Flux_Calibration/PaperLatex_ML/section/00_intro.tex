The development of thermo-elastic constitutive models for the simulation of materials undergoing external mechanical and thermal loading has been the focus of intensive study in numerous References \cite{Betsch2018Thermo,Conde2017}. These constitutive models are typically based on an invariant representation of the Helmholtz free energy functional, defined in terms of the deformation gradient tensor $\vect{F}$ (through the objective right Cauchy-Green tensor $\vect{C}$) and the \textit{absolute temperature} $\theta$ (temperature in the sequel). Based on this energy (potential) functional, many authors \cite{Betsch2018Thermo,Conde2017} have proposed temperature-based consistent implicit energy momentum (EM) time integration schemes for the long term simulation of structural components governed by this Helmholtz potential. This type of time integration schemes are well-known for their robustness and stability properties due to their structure-preserving features, making them ideal for long term stable simulations. We purposely use the term \emph{temperature-based} to stress the fact that this type of algorithms rely on the temperature field as the thermodynamical state variable and, as such, this temperature represents one of the unknowns to be solved. 

EM schemes are regarded as both elegant and robust because they are endowed by construction with the discrete analogue of the conservation properties of the continuum, namely the conservation of total energy, linear and angular momenta. The \textit{consistency} of EM schemes refers to their ability to preserve (or dissipate for non-reversible constitutive models) the total energy of a system in agreement with the laws of thermodynamics \cite{Betsch_EM_viscoelasticity_2010,Betsch_EM_mixed_2017,Betsch2018Thermo,Conde2017}. Consistency of these methods is attained by replacing the (exact) partial derivatives of the Helmholtz free energy functional with respect to its arguments ($\vect{C}$ and $\theta$) with their carefully designed algorithmic counterparts. These algorithmic partial derivatives, also known as \textit{discrete derivatives} \cite{Betsch2018Thermo,Betsch_EM_mixed_2017,EM_Electro_1,Gonzalez_EM_2000,Simo_EM_1992}, are formulated in compliance with the so-called \textit{directionality property} \cite{Gonzalez_EM_2000}. Unfortunately, the success of current temperature-based EM consistent integrators rely on the introduction of discrete derivatives incorporating consistency restoring terms \cite{Hesch_Betsch_EM_thermo_2011} which are not that straightforward to be systematised and generalised, especially if interested in pursuing multi-physics applications beyond thermo-mechanics. 

An alternative approach uses the entropy-based GENERIC framework,
where the entropy $\eta$ is considered as the thermodynamical state variable (see the works of Romero \cite{Romero2010} and Conde \cite{Conde2017}), where EM schemes have been also successfully developed. However, a well-accepted difficulty of entropy-based formulations is the non-trivial consideration of temperature boundary conditions. Indeed, unless relatively simple thermo-mechanical models are considered, a (computationally expensive) Newton-Raphson type procedure is required in order to solve a nonlinear equation relating $\eta$ and $\theta$ on the part of the boundary subjected to prescribed temperature (at each boundary quadrature point). This is one of the main reasons to prefer the Helmholtz free energy functional as a thermodynamic potential over the internal energy and, thus, the temperature over the entropy as the thermodynamical state variable. 

Very recently, Franke et. al. \cite{Betsch2018Thermo} have proposed a novel EM scheme in the context of thermo-elasticity, by taking advantage of the concept of isothermal polyconvexity \cite{Ball_1976,Ball_1976_2,Ball_2002,Ball_Murat_1976,Schroderbook} and the use of a novel tensor cross product pioneered by de Boer \cite{deBoer_book} and re-discovered in the context of nonlinear continuum mechanics by Bonet et al. \cite{Bonet_cross_product_2016,Bonet_polyconvexity_2015,Bonet_hyperbolicity_partI_2015}. In essence, the authors propose the consideration of four discrete derivatives which are used to form algorithmic versions of the second Piola-Kirchoff stress tensor and the entropy of the system. In addition to the discrete derivative with respect to the temperature, three further discrete derivatives are presented, which represent the algorithmic counterparts of the work conjugates of the right Cauchy-Green deformation tensor, its co-factor and its determinant. This strategy leads to simplified expressions of the algorithmic second Piola-Kirchoff stress tensor and entropy, when comparing against those obtained following the classical approach \cite{Gonzalez_EM_2000}. Finally, as the EM scheme in \cite{Betsch2018Thermo} relies on the \textit{re-definition} of the internal energy of the system in terms of the temperature, this ultimately entails a certain level of complexity in the derivation of the discrete derivatives (refer to \ref{sec:properties directionality}).   

The aim of the current paper is the development of a new polyconvexity inspired temperature-based EM scheme which uses the Helmholtz free energy functional as the fundamental thermodynamical potential. As a result, the new EM scheme will be shown to inherit the advantages of that of \cite{Betsch2018Thermo} (i.e. consistency, stability, conservation) whilst, in addition, resulting in dramatically far simpler algorithmic expressions. We conceive the simplification brought forward by the new EM scheme as a crucial preliminary step in order to bridge the gap with recently published EM schemes developed by the authors in the context of electro-elasticity \cite{EM_Electro_1} and, therefore, seek extension to thermo-electro-elasticity.

The outline of this paper is as follows: in Section \ref{sec:nonlinear continuum electromechanics}, some basic principles of kinematics are presented. 
%The definition of the co-factor and the Jacobian will be based on the tensor cross product operation defined in de Boer \cite{deBoer_book}. 
The governing equations in nonlinear thermo-elastodynamics are also presented in this section. Section \ref{sec:constitutive equations} introduces the Helmholtz free energy functional and the internal energy functional. 
Section \ref{sec:dynamics} presents the weak forms associated with the governing equations in thermo-elastodynamics. These will help introducing the new temperature-based one-step implicit EM time integrator scheme for thermo-elastodynamics in Section \ref{sec:time integrator}. 
Section \ref{sec:FEM residuals} briefly describes the finite element implementation of the new time integrator scheme and Section \ref{sec:numerical examples} presents a series of numerical examples in order to validate the conservation properties and robustness of the new scheme. 
Finally, Section \ref{sec:conclusions} provides some concluding remarks. 
\ref{sec:entropy-based formulation} presents an entropy-based EM scheme, counterpart of the temperature-based scheme pursued in this paper. \ref{sec:properties directionality} outlines the definition of the discrete derivative expressions featuring in the proposed time integrator in Section \ref{sec:time integrator}. \ref{sec:Betsch formulation} summarises the EM scheme in Reference \cite{Betsch2018Thermo}, illustrating the differences between this and the new EM proposed. \ref{sec:constitutive model appendix} presents  the steps that need to be carried out in order to derive the thermo-elastic constitutive model presented in Section \ref{sec:constitutive models}.

\noindent \textit{Notation:} Throughout this paper, $\vect{A}:\vect{B}=A_{iI}B_{iI}$, $\forall \vect{A},\vect{B}\in\mathbb{R}^{3\times 3}$,  and the use of repeated indices implies summation. The tensor product is denoted by $\otimes$ and the second order identity tensor by $\vect{I}$. The tensor cross product operation $\Cross$ between two artibrary second order tensor $\vect{A}$ and $\vect{B}$ entails $[\vect{A}\Cross\vect{B}]_{iI}=\mathcal{E}_{ijk}\mathcal{E}_{IJK}A_{jJ}B_{kK}$.
Furthermore, $\vect{\mathcal{E}}$ represents the third-order alternating tensor. Lower case indices $\{i,j,k\}$ will be used to represent the spatial configuration whereas capital case indices $\{I,J,K\}$ will be used to represent the material configuration.
The full and special orthogonal groups in $\mathbb{R}^3$ are represented as $\text{O}(3)=\{\vect{A}\in\mathbb{R}^{3\times 3},\vert\,\vect{A}^T\vect{A}=\vect{I}\}$ and $\text{SO}(3)=\{\vect{A}\in\mathbb{R}^{3\times 3},\vert\,\vect{A}^T\vect{A}=\vect{I},\,\text{det}\vect{A}=1\}$, respectively and the set of invertible second order tensors with positive determinant is denoted by $\text{GL}^+(3)=\{\vect{A}\in\mathbb{R}^{3\times 3} \vert  \,\text{det}{\vect{A}}>0\}$. Differentiation with respect to time of any field ${(\bullet)}$ will be denoted through $\dot{(\bullet)}$.
%\input{section/01_Intro/Intro_v2}
%%

