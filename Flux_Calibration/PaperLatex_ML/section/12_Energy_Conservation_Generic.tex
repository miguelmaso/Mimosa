\begin{equation}\label{eqn:weak forms for the dynamic formulation Dirichlet bcs appendix}
\begin{aligned}
\widetilde{\mathcal{W}}_{\vect{v}} & = \int_{\mathcal{B}_0}\left(\dot{\vect{\varphi}} - \vect{v}\right)\cdot\dot{\vect{v}}\,dV;\\
%
\widetilde{\mathcal{W}}_{\vect{\varphi}} & =  \int_{\mathcal{B}_0}\rho_0\dot{\vect{v}}\cdot\dot{\vect{\varphi}}\,dV + \int_{\mathcal{B}_0}\left(\vect{\Sigma_C} + \vect{\Sigma_G}\Cross\vect{C} + \Sigma_C\vect{G}\right):\dot{\vect{C}}\,dV\\&-  \int_{\mathcal{B}_0}\vect{f}_0\cdot\dot{\vect{\varphi}}\,dV-
\int_{\partial_{t}\mathcal{B}_0}\vect{t}_0\cdot\dot{\vect{\varphi}}\,dA - \int_{\partial_{\vect{\varphi}}\mathcal{B}_0}\vect{PN}\cdot\dot{\bar{\vect{\varphi}}}\,dA;\\
%
\widetilde{\mathcal{W}}_{\varphi} & =  \int_{\mathcal{B}_0}\vect{D}_0\cdot\vect{\nabla}_0\dot{\varphi}\,dV+  \int_{\mathcal{B}_0}\rho_0^e\dot{\varphi}\,dV+
\int_{\partial_{\omega}\mathcal{B}_0}\omega^e_0\dot{\varphi}\,dA - \int_{\partial_{\varphi}\mathcal{B}_0}\vect{D}_0\cdot\vect{N}\dot{\bar{\varphi}}\,dA;\\
%
\widetilde{\mathcal{W}}_{\vect{D}_0} & =   \int_{\mathcal{B}_0}\dot{\vect{D}}_0\cdot \left(\vect{\Sigma}^{\text{sym}}_{\vect{D}_0} + \vect{\nabla}_0\varphi\right)\,dV.
%- \int_{\mathcal{B}_0}\delta\vect{D}_0\cdot\frac{\partial \vect{\nabla}_0\varphi\right)\,dV
%
\end{aligned}
\end{equation}


Notice that the external contributions in above equation \eqref{eqn:weak forms for the dynamic formulation Dirichlet bcs appendix} can be written
%
\begin{equation}\label{eqn:DPiextr appendix}
\begin{aligned}
\int_{\mathcal{B}_0}\vect{f}_0\cdot\dot{\vect{\varphi}}\,dV + 
\int_{\partial_t\mathcal{B}_0}\vect{t}_0\cdot\dot{\vect{\varphi}}\,dA & = \dot{\Pi}^m_{\text{ext}}\left(\dot{\vect{\varphi}}\right) - \int_{\mathcal{B}_0}\dot{\vect{f}}_0\cdot\vect{\varphi}\,dV- \int_{\partial_t\mathcal{B}_0}\dot{\vect{t}}_0\cdot\vect{\varphi}\,dA;\\
%
\int_{\mathcal{B}_0}{\rho}^e_0\dot{{\varphi}}\,dV + 
\int_{\partial_{\omega}\mathcal{B}_0}\omega^e_0\dot{\varphi}\,dA & = \dot{\Pi}^e_{\text{ext}}\left(\dot{{\varphi}}\right) - \int_{\mathcal{B}_0}\dot{\rho}_0^e{\varphi}\,dV- \int_{\partial_t\mathcal{B}_0}\dot{\omega}^e_0{\varphi}\,dA.
%
%
\end{aligned}
\end{equation}

Making use of above equation \eqref{eqn:DPiextr}, the last three terms in both equations \eqref{eqn:weak forms for the dynamic formulation Dirichlet bcs appendix}$_a$ and \eqref{eqn:weak forms for the dynamic formulation Dirichlet bcs appendix}$_b$ can be conveniently written as
%
\begin{equation}\label{eqn:equivalent expression for last three external terms appendix}
\begin{aligned}
-\int_{\mathcal{B}_0}\vect{f}_0\cdot\dot{\vect{\varphi}}\,dV
-\int_{\partial_t\mathcal{B}_0}\vect{t}_0\cdot\dot{\vect{\varphi}}\,dA - \int_{\partial_{\vect{\varphi}}\mathcal{B}_0}\vect{PN}\cdot\dot{\bar{\vect{\varphi}}}\,dA& = -\dot{\Pi}^m_{\text{ext}}\left(\dot{\vect{\varphi}}\right) + \mathcal{I}^m_{\text{ext}}\left(\vect{\varphi}\right);\\
%
%
-\int_{\mathcal{B}_0}{\rho}^e_0\dot{\varphi}\,dV
-\int_{\partial_{\omega}\mathcal{B}_0}{\omega}^e_0\dot{\varphi}\,dA - \int_{\partial_{\varphi}\mathcal{B}_0}\vect{D}_0\cdot\vect{N}\dot{\bar{\varphi}}\,dA& = -\dot{\Pi}^e_{\text{ext}}\left(\dot{\varphi}\right) + \mathcal{I}^e_{\text{ext}}\left(\varphi\right),
%
\end{aligned}
\end{equation}
%
where the functionals $\mathcal{I}^m_{\text{ext}}\left(\vect{\varphi}\right)$ and $\mathcal{I}^e_{\text{ext}}\left(\varphi\right)$ in \eqref{eqn:equivalent expression for last three external terms} are defined as
%
\begin{equation}\label{eqn:functionals Im and Ie appendix}
\begin{aligned}
\mathcal{I}^m_{\text{ext}}\left(\vect{\varphi}\right) &= \int_{\mathcal{B}_0}\dot{\vect{f}}_0\cdot{\vect{\varphi}}\,dV + 
\int_{\partial_{t}\mathcal{B}_0}\dot{\vect{t}}_0\cdot{\vect{\varphi}}\,dA - \int_{\partial \vect{\varphi}\mathcal{B}_0}\vect{PN}\cdot\dot{\bar{\vect{\varphi}}}\,dA;\\
%
\mathcal{I}^e_{\text{ext}}\left({\varphi}\right) &= -\int_{\mathcal{B}_0}\dot{\rho}^e_0{{\varphi}}\,dV -
\int_{\partial_{\omega}\mathcal{B}_0}\dot{\omega}^e_0{\varphi}\,dA - \int_{\partial_{\varphi}\mathcal{B}_0}\vect{D}_0\cdot\vect{N}\dot{\bar{{\varphi}}}\,dA.
%
\end{aligned}
%
\end{equation}
%

Notice that both functionals $\mathcal{I}^m_{\text{ext}}\left(\vect{\varphi}\right)$ and $\mathcal{I}^e_{\text{ext}}\left({\varphi}\right)$ in \eqref{eqn:functionals Im and Ie appendix} account for the rate of change of energy due to the time variation of external loads $\vect{f}_0$ and $\vect{t}_0$, external electric charges $\rho^e_0$ and $\omega^e_0$ and the Dirichlet boundary conditions (i.e $\bar{\vect{\varphi}}$ and $\bar{\varphi}$). Inserting above equation \eqref{eqn:functionals Im and Ie appendix} into \eqref{eqn:weak forms for the dynamic formulation Dirichlet bcs appendix} leads to
%
\begin{equation}\label{eqn:Wvarphi conservation of energy appendix}
\begin{aligned}
\mathcal{W}_{\vect{\varphi}}& = \dot{K} + \int_{\mathcal{B}_0}\left(\vect{\Sigma_C} + \vect{\Sigma_G}\Cross\vect{C} + \Sigma_C\vect{G}\right):\dot{\vect{C}}\,dV-  \dot{\Pi}^m_{\text{ext}}\left(\vect{\varphi}\right) + \mathcal{I}^m_{\text{ext}}\left(\vect{\varphi}\right) = 0;\\
%
\mathcal{W}_{{\varphi}}& = \int_{\mathcal{B}_0}\vect{D}_0\cdot\vect{\nabla}_0\dot{\varphi}\,dV - \dot{\Pi}^e_{\text{ext}}\left(\varphi\right) + \mathcal{I}^e_{\text{ext}}\left(\varphi\right) = 0;\\
%
\mathcal{W}_{\vect{D}_0}& =  \int_{\mathcal{D}_0}\dot{\vect{D}}_0\cdot\left(\vect{\Sigma}^{\text{sym}}_{\vect{D}_0} + \vect{\nabla}_0\varphi\right)\,dV = 0,
\end{aligned}
\end{equation}
%
where the total kinetic energy of the system $K= \int_{\mathcal{B}_0}\frac{\rho_0}{2}\vect{v}\cdot\vect{v}\,dV$.

\noindent\makebox[\linewidth]{\rule{\textwidth}{0.4pt}}

\noindent \textit{Remark 3.} Let us consider a field $\vect{A}$\footnote{ In general $\vect{A}$ could be a second order tensor, a vector or a scalar} that depends on the deformation gradient tensor, namely $\vect{A} = \vect{A}\left(\vect{F}\right)$. The directional derivative of $\vect{A}$ with respect to $\dot{\vect{\varphi}}$ and its time derivative are
%
\begin{equation}
D\vect{A}[\dot{\vect{\varphi}}] = \frac{\partial\vect{A}}{\partial\vect{F}}:D\vect{F}[\delta\dot{\vect{\varphi}}];\qquad
\dot{\vect{A}} = \frac{\partial\vect{A}}{\partial\vect{F}}:\dot{\vect{F}} = \frac{\partial\vect{A}}{\partial\vect{F}}:\vect{\nabla}_0\vect{v}.
\end{equation}

Replacing $\delta\vect{\varphi}$ with $\dot{\vect{\varphi}}$ in \eqref{eqn:directional derivative of F} leads to
$D\vect{F}[\dot{\vect{\varphi}}] = \vect{\nabla}_0\vect{v}$, 
which enables to conclude that $D\vect{A}[\dot{\vect{\varphi}}] = \dot{\vect{A}}$. The same can be concluded for the set of symmetric strain measures $\{\vect{C},\vect{G},C\}$ \eqref{eqn:Cofactor and Jacobian symmetric}, namely
%
\begin{equation}\label{eqn:Cdot Gdot Cdot appendix}
D\vect{C}[\dot{\vect{\varphi}}] = \dot{\vect{C}};\qquad
D\vect{G}[\dot{\vect{\varphi}}] = \dot{\vect{G}} = \vect{C}\Cross\dot{\vect{C}};\qquad
DC[\dot{\vect{\varphi}}] = \dot{C} = \vect{G}:\dot{\vect{C}}.
\end{equation}

The result in \eqref{eqn:Cdot Gdot Cdot appendix} proves the equivalence of the second term on the right-hand side of \eqref{eqn:weak forms for the dynamic formulation Dirichlet bcs appendix} and \eqref{eqn:Wvarphi conservation of energy appendix} when written in terms of  $\{D\vect{C}[\dot{\vect{\varphi}}],D\vect{G}[\dot{\vect{\varphi}}],D{C}[\dot{\vect{\varphi}}]\}$ or in terms of  $\{\dot{\vect{C}},\dot{\vect{G}},\dot{C}\}$.

\noindent\makebox[\linewidth]{\rule{\textwidth}{0.4pt}}

\begin{equation}\label{eqn:Wvarphi modified appendix}
\begin{aligned}
 \dot{\Pi}^m_{\text{ext}}\left(\vect{\varphi}\right) 
- \mathcal{I}^{m}_{\text{ext}}\left(\vect{\varphi}\right) & = \dot{K} + \int_{\mathcal{B}_0}\left(\vect{\Sigma_C} + \vect{\Sigma_G}\Cross\vect{C} + \Sigma_C\vect{G}\right):\dot{\vect{C}}\,dV   + \int_{\mathcal{B}_0}\vect{\Sigma}^{\text{sym}}_{\vect{D}_0}\cdot\dot{\vect{D}}_0\,dV - \int_{\mathcal{B}_0}\vect{\Sigma}^{\text{sym}}_{\vect{D}_0}\cdot\dot{\vect{D}}_0\,dV \\
%
%
&=\dot{K} + \int_{\mathcal{B}_0}\dot{W}_{\text{sym}}\left(\vect{C},\vect{G},C,\vect{D}_0\right)\,dV  - \int_{\mathcal{B}_0}\vect{\Sigma}^{\text{sym}}_{\vect{D}_0}\cdot\dot{\vect{D}}_0\,dV \\
%
&=\dot{K} + \int_{\mathcal{B}_0}\dot{W}_{\text{sym}}\left(\vect{C},\vect{G},C,\vect{D}_0\right)\,dV  + \int_{\mathcal{B}_0}\vect{\nabla}_0\varphi\cdot\dot{\vect{D}}_0\,dV .
%
\end{aligned}
\end{equation}

Integration by parts
%
\begin{equation}
\int_{\mathcal{B}_0}\vect{\nabla}_0\varphi\cdot\dot{\vect{D}}_0\,dV = 
%
-\int_{\partial_{\omega}\mathcal{B}_0}\varphi\dot{\omega}^e_0\,dA - \int_{\mathcal{B}_0}\varphi\dot{\rho}^e_0\,dV =  \mathcal{I}^e_{\text{ext}}\left(\varphi\right).
%
\end{equation}


Hence,
%
\begin{equation}
\dot{\Pi}_{\text{ext}}^m - \mathcal{I}^m_{\text{ext}}\left(\vect{\varphi}\right) -
\mathcal{I}^e_{\text{ext}}\left({\varphi}\right) = \dot{K} + \int_{\mathcal{B}_0}\dot{W}_{\text{sym}}\left(\vect{C},\vect{G},C,\vect{D}_0\right)\,dV.
\end{equation}

It is therefore clear that in the case when $\vect{f}_0$, $\vect{t}_0$, $\rho^e_0$ and $\omega^e_0$ do not depend on time, the following condition holds
%
\begin{equation}\label{eqn:conservation 1 appendix}
\frac{d}{dt}\mathcal{H}^m = 0;\qquad
\mathcal{H}^m =K + \int_{\mathcal{B}_0}W_{\text{sym}}\left(\vect{C},\vect{G},C,\vect{D}_0\right)\,dV - \Pi^m_{\text{ext}}\left(\vect{\varphi}\right),
\end{equation}
%
and therefore the scalar field $\mathcal{H}^m$ is preserved throughout the motion of the EAP and where $\pi_{\text{ext}}$ is the ...

\noindent\makebox[\linewidth]{\rule{\textwidth}{0.4pt}}

\noindent \textit{Remark 4.} Adding $\mathcal{W}_{\vect{\varphi}}$ in \eqref{eqn:Wvarphi modified}, $\mathcal{W}_{\varphi}$ in \eqref{eqn:Wvarphi conservation of energy appendix}$_b$ and $\mathcal{W}_{\vect{D}_0}$ in \eqref{eqn:Wvarphi conservation of energy appendix}$_c$ leads to the following identity
%
\begin{equation}\label{eqn:energy I appendix}
\begin{aligned}
\dot{\Pi}_{\text{ext}}^m + \dot{\Pi}_{\text{ext}}^e - \mathcal{I}^m_{\text{ext}}\left(\vect{\varphi}\right) - 
\mathcal{I}^e_{\text{ext}}\left({\varphi}\right)& =
%\dot{K} + \int_{\mathcal{B}_0}\dot{W}_{\text{sym}}\,dV  - \int_{\mathcal{B}_0}\vect{\Sigma}^{\text{sym}}_{\vect{D}_0}\cdot\dot{\vect{D}}_0\,dV +
%
%\int_{\mathcal{B}_0}\vect{D}_0\cdot\vect{\nabla}_0\dot{\varphi}\,dV+
%
%\int_{\mathcal{B}_0}\dot{\vect{D}}_0\cdot\left(\vect{\Sigma}^{\text{sym}}_{\vect{D}_0} + \vect{\nabla}_0\varphi\right)\,dV \\
%
%&
%=
\dot{K} + \int_{\mathcal{B}_0}\dot{W}_{\text{sym}}\,dV  +
%
\int_{\mathcal{B}_0}\vect{D}_0\cdot\vect{\nabla}_0\dot{\varphi}\,dV+
%
\int_{\mathcal{B}_0}\dot{\vect{D}}_0\cdot\vect{\nabla}_0\varphi\,dV\\
%
&=\dot{K} + \int_{\mathcal{B}_0}\dot{W}_{\text{sym}}\,dV  -
%
\int_{\mathcal{B}_0}\vect{D}_0\cdot\dot{\vect{\Sigma}}^{\text{sym}}_{\vect{D}_0}\,dV-
%
\int_{\mathcal{B}_0}\dot{\vect{D}}_0\cdot\vect{\Sigma}^{\text{sym}}_{\vect{D}_0}\,dV\\
%
&=\dot{K} + 
%
\int_{\mathcal{B}_0}\frac{d}{dt}\left({W}_{\text{sym}} - \vect{D}_0\cdot{\vect{\Sigma}}^{\text{sym}}_{\vect{D}_0}\right)\,dV,
%
\end{aligned}
\end{equation}
%
where since $\vect{D}_0$, $\dot{\vect{D}}_0$ belong to the same functional space $\mathbb{V}^{\vect{D}_0}$, by virtue of equation \eqref{eqn:Wvarphi conservation of energy appendix}$_c$ the first term on the right-hand side of \eqref{eqn:energy I appendix} into its equivalent expression on the .... Making use of equation  \eqref{eqn:Legendre transform} it is possible to re-write \eqref{eqn:energy I appendix} as
%
\begin{equation}\label{eqn:energy II appendix}
\begin{aligned}
\dot{\Pi}_{\text{ext}}^m + \dot{\Pi}_{\text{ext}}^e - \mathcal{I}^m_{\text{ext}}\left(\vect{\varphi}\right) - \mathcal{I}^e_{\text{ext}}\left(\varphi\right)& =
\dot{K} + 
%
\int_{\mathcal{B}_0}\dot{\varPhi}_{\text{sym}}\left(\vect{C},\vect{E}_0\right)\,dV.
%
\end{aligned}
\end{equation}

It is therefore clear that in the case when $\vect{f}_0$, $\vect{t}_0$, $\rho^e_0$ and $\omega^e_0$ do not depend on time, the following condition holds
%
\begin{equation}\label{eqn:conservation 2 appendix}
\frac{d}{dt}\mathcal{H} = 0;\qquad
\mathcal{H} = K + \int_{\mathcal{B}_0}\varPhi_{\text{sym}}\left(\vect{C},\vect{E}_0\right)\,dV - \Pi^m_{\text{ext}}\left(\vect{\varphi}\right)
-\Pi^e_{\text{ext}}\left(\varphi\right),
\end{equation}
%
and therefore the scalar field $\mathcal{H}$ is preserved throughout the motion of the EAP. Notice that $\mathcal{H}$ is the Hamiltonian of the system, defined through the following Legendre transformation
%
\begin{equation}\label{eqn:legendre transform for the Hamiltonian appendix}
\mathcal{H}\left(\vect{x},\vect{p},\varphi,\vect{D}_0\right) = \sup_{\vect{p}}\left\{\int_{\mathcal{B}_0}\vect{p}\cdot\vect{v}\,dV - \int_{\mathcal{B}_0}\mathcal{L}\left(\vect{\varphi},\dot{\vect{\varphi}},\varphi,\vect{D}_0\right)\,dV +  \Pi^m_{\text{ext}}\left(\vect{\varphi}\right) + 
\Pi^e_{\text{ext}}\left(\varphi\right)\right\},
\end{equation}
%
where $\vect{p} = \rho_0\vect{v}$ in \eqref{eqn:legendre transform for the Hamiltonian appendix} represents the linear momentum per unit undeformed volume.

\noindent\makebox[\linewidth]{\rule{\textwidth}{0.4pt}}
