


A new one-step implicit and thermodynamically consistent Energy-Momentum (EM) preserving time integration scheme has been presented for the simulation of structural components undergoing large deformations and temperature fields, for which well-posed constitutive models are used for the entire range of deformations and temperature. The use of polyconvexity inspired constitutive models and the new tensor cross product algebra pioneered by de Boer \cite{deBoer_book} and re-discovered by Bonet et al. \cite{Bonet_cross_product_2016} in the context of nonlinear solid mechanics, are key for the development of the discrete derivatives, fundamental for the construction of the EM algorithmic derived variables, namely the second Piola-Kirchoff stress tensor and the entropy (or the absolute temperature). The proposed scheme inherits the advantages of the EM scheme recently published by Franke et al. \cite{Betsch2018Thermo} (i.e. consistency, stability, conservation) whilst resulting in dramatically far simpler algorithmic expressions, thus circumventing a bottleneck and paving the way for the incorporation of further physics into the model. A series of numerical examples have been presented in order to demonstrate the robustness and applicability of the new EM scheme. These examples make use of a temperature-based version of the EM scheme (using the Hemlholtz's free energy as the thermodynamical potential and the temperature as the thermodynamical state variable). \ref{sec:entropy-based formulation} includes an entropy-based analogue EM scheme (using the internal energy as the thermodynamical potential and the entropy as the thermodynamical state variable).   



%
%A new one-step implicit Energy-Momentum (EM) time integration scheme has been presented for the simulation of thermo-elastic processes. The proposed EM scheme is inspired by the previous EM scheme published by Betsch et. al. \cite{Betsch_EM_thermo_2017}, whereby taking advantage of the concept of polyconvexity \cite{Ball_1976,Ball_1976_2,Ball_2002,Ball_Murat_1976,Schroderbook}, and extended system of (four) greatly simplified discrete derivatives is proposed. Although the formulation in \cite{Betsch_EM_thermo_2017} is based on temperature, it relies on the definition of a pseudo-internal energy functional which in reality depends on temperature, and not on entropy. This introduces a certain complexity in the formulation (more specifically, in its consistent linearisation) that the present paper aims to minimise. In particular, we present two possible EM schemes, based on  temperature or entropy as the thermodynamic state variable. With that in mind, we use for each scheme the consistent energy functional, namely the Helmholtz free energy functional (for the temperature-based scheme) or the consistent Legendre transformed internal energy  functional (for the entropy-based scheme), respectively. Only the finite element implementation of the temperature-based scheme is pursued in this paper. The design of the entropy-based scheme can be found in \ref{sec:entropy-based formulation}. Three numerical examples have been included in Section \ref{sec:numerical examples} in order to demonstrate the stability and conservation properties of the proposed temperature-based EM scheme.

