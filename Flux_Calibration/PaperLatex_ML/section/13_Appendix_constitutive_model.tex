The objective of this appendix is to briefly recall the calorimetry considerations followed in order to derive the constitutive model presented in Section \ref{sec:constitutive models}. For that, we start by re-expressing the internal energy $\widetilde{e}(\vect{C},\eta)$ as a function of the absolute temperature as
%
\begin{equation}
\hat{e}(\vect{C},\theta) = \widetilde{e}(\vect{C},\eta(\vect{C},\theta)).
\end{equation}
%
%with
%%
%\begin{equation}
%\hat{e}(\vect{C},\theta)=\hat{W}(\vect{C},\vect{G},C,\theta);\qquad
%\widetilde{e}(\vect{C},\eta(\vect{C},\theta))=\widetilde{W}(\vect{C},\vect{G},C,\eta(\vect{C},\vect{G},C,\theta)).
%\end{equation}

Calorimetry principles permit to experimentally measure the change of internal energy as a function of the temperature (for a constant deformation) yielding 
%
\begin{equation}\label{eqn:heat capacity}
\partial_{\theta}\hat{e} = c_v,
\end{equation}
%
with $c_v$ denoting the heat capacity of the material. Notice that above equation \eqref{eqn:heat capacity} can be equivalently written as
%
\begin{equation}\label{eqn:thermo model I}
\partial_{\theta}\hat{e} = \partial_{\eta}\widetilde{e}\,\partial_{\theta}\eta = \theta\partial_{\theta}\eta   =c_v\Rightarrow
%
\partial_{\theta}\eta = \frac{c_v}{\theta}.
\end{equation} 

Integration of \eqref{eqn:thermo model I} results in
%
\begin{equation}\label{eqn:thermo model II}
\begin{aligned}
\int_{\eta(\vect{C},\theta_R)}^{\eta(\vect{C},\theta)}\,d\eta = 
\int_{\theta_R}^{\theta}\frac{c_v}{\theta}\,d\theta\Rightarrow 
%
\eta(\vect{C},\theta) = \eta_R(\vect{C}) + c_v\ln \frac{\theta}{\theta_R},
\end{aligned}
\end{equation}
%
with $\eta_R(\vect{C}):=\eta(\vect{C},\theta_R)$.
Since $\eta=-\partial_{\theta}\widetilde{\Psi}$ we can further integrate \eqref{eqn:thermo model II} as
%
\begin{equation}
\int_{\widetilde{\Psi}(\vect{C},\theta_R)}^{\widetilde{\Psi}(\vect{C},\theta)}\,d\widetilde{\Psi}  = -\int_{\theta_R}^{\theta}\eta(\vect{C},\theta)\,d\theta, 
\end{equation}
%
yielding
%
\begin{equation}
\begin{aligned}
\widetilde{\Psi}(\vect{C},\theta)&= \widetilde{\Psi}(\vect{C},\theta_R)-\int_{\theta_R}^{\theta}\left(\eta_R(\vect{C})+c_v\ln\frac{\theta}{\theta_R}\right)\,d\theta\\
%
&=\widetilde{\Psi}(\vect{C},\theta_R)-(\theta-\theta_R)\eta_R(\vect{C}) +  c_v\left(\theta-\theta_R - \theta\ln\frac{\theta}{\theta_R}\right),
%
\end{aligned}
\end{equation}
%
which can be finally written as
%
\begin{equation}
\begin{aligned}
\widetilde{\Psi}(\vect{C},\theta)&= \widetilde{\Psi}_m(\vect{C}) -{\eta}_R(\vect{C})(\theta-\theta_R)+\widetilde{\Psi}_{\theta}(\theta),
%
\end{aligned}
\end{equation}
%
with
%
\begin{equation}
\begin{aligned}
\widetilde{\Psi}_m(\vect{C}) = \widetilde{\Psi}(\vect{C},\theta_R);\qquad
%
%
\widetilde{\Psi}_{\theta}(\theta)= c_v\left(\theta-\theta_R - \theta\ln\frac{\theta}{\theta_R}\right).
\end{aligned}
%
\end{equation}






